\documentclass{article}
\usepackage[utf8]{inputenc}
\usepackage{fancyhdr}
 
\pagestyle{fancy}
\fancyhf{}
\rhead{Sheriden Goldie - 42611814}
\lhead{FOAR705 - Digital Humanities}
\lfoot{August 2019}
\rfoot{Page \thepage}

\title{Proof of Concept - Scoping Exercise}
\author{By Sheriden Goldie}
\date{}
\begin{document}

\maketitle

\tableofcontents
\pagebreak

\section{Introduction}
This document will address potential avenues for development of technology in the service of my research project. 

I am completing a Master of Research in Creative Writing. This means that my thesis will be made up of a creative component and an exegesis. The creative component will most likely take the form of a short story. An exegesis is a piece of writing that critically engages with my own writing and writing process. 

The exegesis will be based on scholarly research, and will be directly linked to the creative component. The link to the creative task can be in a number of ways, the primary one being critical engagement with ones own work, as well as the integration of research. 

To inform the process of writing a thesis in Creative Writing, my reading is very wide ranging. So when I note reading in my work flow, it could be referring to any combination of the following text types:
\begin{itemize}
    \item Novels/Novellas
    \item Short stories
    \item Poems
    \item Journal articles (inside my field and in other areas of interest e.g. Science)
    \item Non-fiction books (creative and theory based)
    \item Blogs/online content
    \item Interviews
    \item Films (visual product, as well as scripts)
    \item TV Shows/Series (ibid)
\end{itemize}

While I do not work with quantifiable data, I do work with qualitative forms of data, in the form of notes, and critical responses to readings, and my own thoughts, ideas, and creative experimentation.

\section{Work Flow}

\subsection{Prepatory Tasks}
Tasks for preparing for a Creative Writing Thesis:
\begin{itemize}
    \item Pre-reading and note taking
    \item Drafting Research Proposal
    \item Writing exploratory works in field of interest
\end{itemize}

\subsection{Daily Tasks}
Daily tasks for completion of a Creative Writing Thesis:
\begin{itemize}
    \item Reading relevant texts
    \item Recording notes, thoughts, and responses to texts
    \item Compiling notes for short articles/chapters
    \item Creative writing (Either exercise driven, or following on from previous work)
    \item Editing and Rewriting
    \item Maintaining reference and citation notes
\end{itemize}

\subsection{Finalisation and Publishing}
Tasks for finalisation and potential publication of Creative Writing Thesis:
\begin{itemize}
    \item Proofreading
    \item Editing
    \item Formatting to conventions or publication specifications
    \item Reference and citation checking
    \end{itemize}

\section{Pain Points}
The most important and valuable work I can do on my thesis is cognitive. It is important that I can carefully think through my ideas, and convey these articulately in my writing. 
At present most of my projects have been small enough that an ad-hoc approach to note maintenance has been sufficient. I am aware that with a larger project - like a combined thesis of
20,000 words - this is not going to be appropriate.

Formatting is also a pinch point for me, as often publications will have different formatting requirements for submission. This is problematic with word processing programs, as changing format to suit specification, often disrupts the contents - especially in the case of poetry. 

Reference and citation management is also an area that up until now has not proved to be an issue. It is possible for me to hold enough information in my head for a 3,000 word essay, and remember where citations came from. I have successfully used online tools to create bibliographies, however the continuing presence of these services cannot be assured. Also different publications have different requirement on citation methods. 

\section{Gain Points}
Creating a way to digitise and index notes on critical reading responses would be an invaluable gain for large projects. This would ideally be search-able, and able to be organised by different priorities. This would enable the note maintenance to be minimised, though it might add a step to the note-taking process depending on the initial method I chose to use. 

With the awareness that online services for bibliography and citation creation are not a guaranteed resource, it would be valuable to have a way to manage and create these in various formats without relying on external services. 

\section{Pain Relievers and Gain Deliverers}
Key issues raised in Pain Points and Gain Points:
\begin{itemize}
    \item Note taking and note management
    \item Formatting, and ability to change format for purpose (without damaging content)
    \item Reference and citation management
\end{itemize}

To address the issue of note taking - I propose to create a form to fill out for each entry, which can then be indexed for sorting and retrieval. This would also include reference information for later citation should the information be used.
This is a tool that will be useful at all stages of thesis creation as well, and it could be valuable to the wider academic community. 

Utilising a product like LaTeX for document creation that will allow for formatting changes without damaging the contents of the document would be a great way to allow for flexibility when submitting various documents. 

A system or program that will manage references and citations, and could potentially link to the notes database would be amazing. This would also be in response to taking control of one's own project from outsourced services. 

\end{document}

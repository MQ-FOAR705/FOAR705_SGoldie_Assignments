\documentclass{article}
\usepackage[utf8]{inputenc}
\usepackage{fancyhdr}
\usepackage{graphicx}
\graphicspath{ {images/} }
\usepackage{hyperref}
%setup for hyperref colour schemes
\hypersetup{
    colorlinks=true,
    linkcolor=blue,
    filecolor=magenta,      
    urlcolor=cyan,
    pdftitle={Sharelatex Example},
    pdfpagemode=FullScreen,}

\usepackage{geometry}
\geometry{a4paper, left=25mm, right=25mm, top=25mm, bottom=25mm}
\setlength{\parindent}{2em}
\setlength{\parskip}{1em}
\renewcommand{\baselinestretch}{1.0}
 

\pagestyle{fancy}
\fancyhf{}
\rhead{Sheriden Goldie - 42611814}
\lhead{FOAR705 - Digital Humanities}
\lfoot{August 2019}
\rfoot{Page \thepage}

\title{Elaboration}
\author{By Sheriden Goldie}
\date{}
\begin{document}

\maketitle

\tableofcontents



\section{Task Outline}
In my scoping documents I outline three overarching processes; data collection, data organisation, and data synthesis.

There is also the aspect of writing and publication - which are aspects that I plan to address as well by utilising existing software/frameworks. 

In articulating these processes in my research, I also decomposed them to smaller steps and processes. This has allowed me to seek out technological applications that might address these steps, and allow them to be done more effectively.

\section{Preliminary Research}
In articulating my process, I was better equipped to ask questions about what kinds of software could begin to address my concerns. I utilised the consultation time with my teachers, and Shawn suggested looking into Content Management Systems (CMS), like Omeka, and archival management software like Tropy. 

In looking up open source CMS software, specifically in relation to documents and literary studies, I found a similar category of software - Document Management System (DMS). I also found a website by Nick Blackbourn, who implemented a digital note taking and management strategy when writing his history PhD. This is an excellent framework for integrating software to a literary research application.

See: https://nickblackbourn.com/blog/digital-note-taking/

This also highlights to me that it is likely that my data collection and data organisation is likely to be managed by a single software. However this leaves me with finding some kind of approach to mapping, or visualising connections.

\section{Technologies}
The following are software that I plan to investigate for their suitability for incorporation into my process. 


\subsection{Data Collection and Organisation}

\begin{itemize}
    \item DevonThink - a propriety DMS. Utilised in Nick Blackbourn's process
    \item Zim - an open source DMS.
    \item OneNote - a propriety note taking software. Owned/operated by Microsoft - but available to me through the use of Windows 10. 
    \item Nebo - a note taking software that interprets handwriting into typed text.
    \item Scrivener - a propriety writing tool that allows various organisational functions
\end{itemize}


\subsection{Data Synthesis and Mapping}

\begin{itemize}
    \item Scrapple - a propriety mind mapping software. This is a program with the specific task of mind mapping - so its ability to interface with other programs is limited.
    \item Tinderbox - a propriety mind mapping and idea organisation software. But it is only compatible with Mac OS as far as I can see.
    \item Miro - an open source mind-mapping/white board software.
    \item there are capabilities within some DMS for tagging/keyword assignment which might fulfil this aspect just enough to be useful.
    \item  Realistically, I'm not sure if something exists that can do what I am imagining. 
\end{itemize}


\subsection{Writing and Publication}

\begin{itemize}
    \item ConTex - we have been using LaTeX through Overleaf. I would like to look into ConTex as an alternative more suited to my desire to have more direct control over formatting.
    \item BibTex - a reference management tool that links well with LaTeX frameworks
    \item Mendeley - reference management service that is available to me through Macquarie University
\end{itemize}


\section{Testing Data Collection and Organisation}

Please refer to the ``Elaboration Testing'' in my Learning Journal via this \href{https://www.overleaf.com/read/vgmsjfphycpq}{ShareLink}


\section{Further Investigation/Testing}

\section{Conclusion and Project Outline}


\end{document}
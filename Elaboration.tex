\documentclass{article}
\usepackage[utf8]{inputenc}
\usepackage{fancyhdr}
\usepackage{graphicx}
\graphicspath{ {images/} }
\usepackage{hyperref}
%setup for hyperref colour schemes
\hypersetup{
    colorlinks=true,
    linkcolor=blue,
    filecolor=magenta,      
    urlcolor=cyan,
    pdftitle={Sharelatex Example},
    pdfpagemode=FullScreen,}

\usepackage{geometry}
\geometry{a4paper, left=25mm, right=25mm, top=25mm, bottom=25mm}
\setlength{\parindent}{2em}
\setlength{\parskip}{1em}
\renewcommand{\baselinestretch}{1.0}
 

\pagestyle{fancy}
\fancyhf{}
\rhead{Sheriden Goldie - 42611814}
\lhead{FOAR705 - Digital Humanities}
\lfoot{August 2019}
\rfoot{Page \thepage}

\title{Elaboration}
\author{By Sheriden Goldie}
\date{}
\begin{document}

\maketitle

\tableofcontents



\section{Task Outline}
In my scoping documents I outline three overarching processes; data collection, data organisation, and data synthesis.

There is also the aspect of writing and publication - which are aspects that I plan to address as well by utilising existing software/frameworks. 

In articulating these processes in my research, I also decomposed them to smaller steps and processes. This has allowed me to seek out technological applications that might address these steps, and allow them to be done more effectively.

\section{Preliminary Research}
In articulating my process, I was better equipped to ask questions about what kinds of software could begin to address my concerns. I utilised the consultation time with my teachers, and Shawn suggested looking into Content Management Systems (CMS), like Omeka, and archival management software like Tropy. 

In looking up open source CMS software, specifically in relation to documents and literary studies, I found a similar category of software - Document Management System (DMS). I also found a website by Nick Blackbourn, who implemented a digital note taking and management strategy when writing his history PhD. This is an excellent framework for integrating software to a literary research application.

See: https://nickblackbourn.com/blog/digital-note-taking/

This also highlights to me that it is likely that my data collection and data organisation is likely to be managed by a single software. However this leaves me with finding some kind of approach to mapping, or visualising connections.

\section{Technologies}
The following are software that I plan to investigate for their suitability for incorporation into my process. 


\subsection{Data Collection and Organisation}

\begin{itemize}
    \item DevonThink - a propriety DMS. Utilised in Nick Blackbourn's process
    \item Zim - an open source DMS.
    \item OneNote - a propriety note taking software. Owned/operated by Microsoft - but available to me through the use of Windows 10. 
    \item Nebo - a note taking software that interprets handwriting into typed text.
    \item Scrivener - a propriety writing tool that allows various organisational functions
\end{itemize}


\subsection{Data Synthesis and Mapping}

\begin{itemize}
    \item Scrapple - a propriety mind mapping software. This is a program with the specific task of mind mapping - so its ability to interface with other programs is limited.
    \item Tinderbox - a propriety mind mapping and idea organisation software. But it is only compatible with Mac OS as far as I can see.
    \item Miro - an open source mind-mapping/white board software.
    \item there are capabilities within some DMS for tagging/keyword assignment which might fulfil this aspect just enough to be useful.
    \item  Realistically, I'm not sure if something exists that can do what I am imagining. 
\end{itemize}


\section{Testing Data Collection and Organisation}

I was unable to test DevonThink and Tinderbox, as these both required purchase, and a macOS to run them, and I do not have anything with macOS.

\subsection{Link to Test Document}

Please refer to the ``POC-ElaborationTesting'' document in my Learning Journal via this \href{https://www.overleaf.com/read/vgmsjfphycpq}{ShareLink}

\section{Conclusion and Further Investigation/Testing}

Through testing these programs I found a number of avenues to explore further. 

Nebo/Interactive Ink is a potentially useful tool for modifying my note taking process, however it is fiddly at present and the benefit of having a digital handwriting recognition, over utilising pen and paper, seems moots. The process of pen and paper notes adds an element of distillation when digitising them for me, which is cognitively beneficial. 

I have not discovered the full utility of Scrivener as both a writing program, and an organisation program. It seems to have the closest fit to the goal I want to achieve. The thing it is missing is tagging to create a network of meaning.
But its organisational flexibility is great, and being able to have a document open to view and a window open to type is very useful, and less clunky than having two resized windows on a screen. 

Scapple also seemed to have the unique possibility of importing data. It was able to receive a text document, and read csv style data. This opens a specific avenue I can pursue when investigating Scrivener further. As these two products are from the same company as well, it may be possible to get further support either from the company, or the community that uses these programs. 

Making Scapple visually appealing is more difficult and tedious - but aesthetic formatting doesn't necessarily add to the end result enough to make Miro a better choice to explore for the purpose of finding a way to link the research output to a visual mapping tool.

Miro is an exceptional tool that I will keep an eye on for other projects, but not this particular pipeline. It has great potential for presentations and organising groups, as well as mapping ideas. 

\end{document}
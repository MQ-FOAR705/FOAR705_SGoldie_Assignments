\documentclass{article}
\usepackage{geometry}
\usepackage[utf8]{inputenc}
\usepackage{graphicx}
% package and file page setting for image input
\graphicspath{ {images/} }

%Allows for linking within document and externally
\usepackage{hyperref}
%setup for hyperref colour schemes
\hypersetup{
    colorlinks=true,
    linkcolor=blue,
    filecolor=magenta,      
    urlcolor=cyan,
    pdftitle={Sharelatex Example},
    pdfpagemode=FullScreen,
    }

\usepackage{geometry}
% package for setting page format and size, and setting paragraph indents, paragraph spacing, and line spacing. 
\geometry{a4paper, left=25mm, right=25mm, top=25mm, bottom=25mm}
\setlength{\parindent}{2em}
\setlength{\parskip}{1em}
\renewcommand{\baselinestretch}{1.3}
% for Baseline stretch
% 1 = 1 line spacing in Word 
% 1.3 = 1.5 line spacing in Word 
% 1.5 = double spacing in Word

\usepackage{fancyhdr}
%package to include header and footer information. Only starts from second page with these settings.

\pagestyle{fancy}
\fancyhf{}
\rhead{Sheriden Goldie - 42611814}
\lhead{FOAR705 - Digital Humanities}
\lfoot{Session 2, 2019}
\rfoot{Page \thepage}


\title{Proof of Concept Design}
\author{Sheriden Goldie}
\date{Session 2, 2019}

\begin{document}

\maketitle

\tableofcontents
%creates a table of contents based on sections and subsections

\pagebreak
%ends text on current page, and moves all following text to the next page.

\section{Creating Project Objectives}
\subsection{User Stories}

\textbf{Data Collection}
\begin{itemize}
    \item As an MRes student writing a thesis, I want to be able to store research data digitally as well as the relevant metadata (citation information) so that my writing process is more efficient.
    \item As an MRes student writing a thesis, I want to be able to keep notes digitally, and be able to tag digital material in a way that is searchable.
    \item As an MRes student writing a thesis, I want to keep relevant source extracts digitally to allow me to integrate these easily when writing my thesis draft. 
\end{itemize}

\noindent \textbf{Data Organisation}
\begin{itemize}
    \item As an MRes Student writing a thesis, I want to be able to organise my sources, quotes, notes, and thesis drafts in a way that is convenient (all in one place) and meaningful (structured logically, and able to be adapted for other projects).
    \item As an MRes student writing a thesis, I want my data organisation system to have a hierarchy and be manipulated to allow ease of focusing on one part/section at a time, to make working on multiple aspects straightforward.
    \item As an MRes student writing a thesis, I want the organisational system I use to be understood by someone other than myself (e.g., my supervisor) so that my research can be transparent, and the system shareable.
\end{itemize}

\subsection{Acceptance Criteria}
\textbf{Data Collection}
\begin{enumerate}
    \item As an MRes student writing a thesis, I should be able to (DC1.1) store source material digitally, (DC1.2) including metadata required to create a reference or citation.
    \item As an  MRes student writing a thesis, I should be able to (DC2.1) place digital documents in an organisational system, and (DC2.2) be able to search within that system for tagged keywords. 
    \item As an MRes student writing a thesis, I should be able to (DC3.1) view and access multiple documents easily while writing.
\end{enumerate}
    


\noindent \textbf{Data Organisation}
\begin{enumerate}
    \item As an MRes student writing a thesis, I should be able to to (DO1.1)export my organisational structure as a template for other research projects. 
    \item As an MRes student writing a thesis, I should be able to (DO2.1) expand and collapse my organisational structure according to logical hierarchies.
    \item As an MRes student writing a thesis (DO3.1) the organisational system I use will be understandable to someone else, and (DO3.2) the structure will be shareable to other students.
\end{enumerate}

\subsection{Themes} 

Key themes in these acceptance criteria:
This is also the order in which things need to be completed/confirmed as following items require the previous one to be working before testing.

\begin{itemize}
    \item Storing data digitally (DC1.1)
    \item Storing searchable metadata (DC1.2, DC2.2)
    \item Organising data digitally (DO2.1)
    \item Exporting data (DC1.2, DO3.2)
    \item Making the structure understandable and transferable (DO3.1, DO3.2)
    \item Sharing organisational structure (DO1.1)
\end{itemize}

\subsection{Prerequisites}


As I plan to utilise Scrivener for the majority of the delivery of this project, having this software is a key prerequisite for all aspects of these user stories. 

I have access to a licensed copy on my home computer - but the trial period for the full software is 30 non-consecutive days which is what I will utilise for some aspects of this delivery. 

\section{Project Management}

For project management, I have selected Trello as an online tool.

Trello seems to be widely used in professional contexts, and is based on a widely used `kaban' (Japanese signboard) method. This means that learning and using this system will be beneficial beyond this particular project.

Link to Trello board: \href{https://trello.com/b/uV1QStsk}{https://trello.com/b/uV1QStsk}


\end{document}

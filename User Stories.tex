\documentclass{article}
\usepackage{geometry}
\usepackage[utf8]{inputenc}
\usepackage{graphicx}
% package and file page setting for image input
\graphicspath{ {images/} }

%Allows for linking within document and externally
\usepackage{hyperref}
%setup for hyperref colour schemes
\hypersetup{
    colorlinks=true,
    linkcolor=blue,
    filecolor=magenta,      
    urlcolor=cyan,
    pdftitle={Sharelatex Example},
    pdfpagemode=FullScreen,
    }

\usepackage{geometry}
% package for setting page format and size, and setting paragraph indents, paragraph spacing, and line spacing. 
\geometry{a4paper, left=25mm, right=25mm, top=25mm, bottom=25mm}
\setlength{\parindent}{2em}
\setlength{\parskip}{1em}
\renewcommand{\baselinestretch}{1.3}
% for Baseline stretch
% 1 = 1 line spacing in Word 
% 1.3 = 1.5 line spacing in Word 
% 1.5 = double spacing in Word

\usepackage{fancyhdr}
%package to include header and footer information. Only starts from second page with these settings.

\pagestyle{fancy}
\fancyhf{}
\rhead{Sheriden Goldie - 42611814}
\lhead{FOAR705 - Digital Humanities}
\lfoot{Session 2, 2019}
\rfoot{Page \thepage}


\title{Proof of Concept Design}
\author{Sheriden Goldie}
\date{Session 2, 2019}

\begin{document}

\maketitle

\tableofcontents
%creates a table of contents based on sections and subsections

\pagebreak
%ends text on current page, and moves all following text to the next page.

\section{Creating Project Objectives}
\subsection{User Stories}

\textbf{Data Collection}
\begin{itemize}
    \item As an MRes student writing a thesis, I want to be able to collect research data digitally and connect it to relevant metadata (source information) so that my writing process is more efficient.
    \item As an MRes student writing a thesis, I want to be able to keep notes digitally, and be able to digital material in a way that is searchable.
\end{itemize}

\noindent \textbf{Data Organisation}
\begin{itemize}
    \item As an Mres Student writing a thesis, I want to be able to organise my sources, quotes, notes, and thesis drafts in a way that is convenient (all in one place) and meaningful (structured logically, and able to be adapted for other projects).
    \item As an Mres student writing a thesis, I want the organisational system I use to be understood by someone other than myself (eg, my supervisor) so that my research can be transparent, and the system shareable.
\end{itemize}

\subsection{Acceptance Criteria}
\textbf{Data Collection}
\begin{itemize}
    \item As an Mres student writing a thesis, I should be able to (1) store source material digitally, (2) including metadata required to create a reference or citation.
    \item As an  Mres student writing a these, I should be able to (1) keep notes digitally, and be able to (2) place those notes in an organisational system, and (3) be able to search within that system for keywords. 
\end{itemize}

\noindent \textbf{Data Organisation}
\begin{itemize}
    \item As an Mres student writing a thesis, I should be able to (1) organise (1.1) sources, (1.2) quotes, (1.3) notes, and (1.4) thesis drafts in one place, and (2) in a system that is structured logically, and (3) I should be able to export this organisational structure as a template for other research projects.
    \item As an Mres student writing a thesis (1) the organisational system I use will be understandable to someone else, and (2) the structure will be shareable to other students.
\end{itemize}


\subsection{Themes and Prerequisites}


\section{Project Management}

For project management, I have selected Trello as an online tool.

Trello seems to be widely used in professional contexts, and is based on a widely used `kaban' (Japanese signboard) method. This means that learning and using this system will be beneficial beyond this particular project.

\end{document}

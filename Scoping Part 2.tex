\documentclass{article}
\usepackage{geometry}
\usepackage[utf8]{inputenc}
\usepackage{graphicx}
\graphicspath{ {images/} }

\usepackage{geometry}
\geometry{a4paper, left=25mm, right=25mm, top=25mm, bottom=25mm}
\setlength{\parindent}{2em}
\setlength{\parskip}{1em}
\renewcommand{\baselinestretch}{1.3}


\usepackage{fancyhdr}
\pagestyle{fancy}
\fancyhf{}
\rhead{Sheriden Goldie - 42611814}
\lhead{FOAR705 - Digital Humanities}
\lfoot{Session 2 2019, Macquarie University}
\rfoot{Page \thepage}

\title{Proof of Concept - Scoping Exercise Part 2}
\author{By Sheriden Goldie}
\date{}

\begin{document}

\maketitle

\tableofcontents
%creates a table of contents based on sections and subsections

\pagebreak
%ends text on current page, and moves all following text to the next page.

\section{Identification}
In the process of completing a Business Analysis (Scoping Exercise 1) on my research process, I identified some key pain areas for further investigation.

These were:
\begin{itemize}
    \item Note taking and note management
    \item Formatting, and ability to change format for purpose (without damaging content)
    \item Reference and citation management
\end{itemize}

\subsection{Problem Identification}
The largest and most time consuming of these pains is ``Note taking and note management.''

The key aspects of this problem are:
\begin{itemize}
    \item Recording notes in a meaningful and organizable way
    \item Documentation of notes - recording note metadata (that being the source material, page numbers, date recorded etc)
    \item Recording and organizing personal notes/responses to other notes
    \item creating networks of meaning from notes - mapping the key terms that link notes (or identifying notes that don't link to others)
    \item being able to access these notes and their metadata easily during final composition to enable to process of writing to be more flowing.
\end{itemize}

Even as I write these notes on the problem I can see there are certain points that need to be addressed in expressing this problem. 
\begin{itemize}
    \item What do I mean by a ``note''?
    \item Is there terminology that can express my ideas better?
    \item What are the different layers of data that I will be working with?
\end{itemize}

\section{Clarification}
\subsection{Key Project Terms}

To clarify this problem I am proposing to use the following terms:
\begin{itemize}
    \item Source text - the written document that information is taken from directly
    \item Quote - a direct transcription from a source text
    \item Source Text Metadata - the information necessary to identify the source text - enabling another person to find that text. Organised in such a way as to be useful for creating in-text citations and reference lists.
    \item Note - text written by the researcher reflecting on the content of a quote
    \item Tag - a unique category of metadata that is attached to a Quote or Note that enables it to be linked to other quotes or notes. Being able to render these links visually wold be nice. 
\end{itemize}

\section{Resolution Goal}
Considering this problem in terms of computational thinking, I will attempt to apply a practice of Decompostion, Pattern Recognition, and Algorithm Design. Though at this stage the latter practice of Algorithm design will not be a fully realised articulation.

\subsection{Decomposition}
Breaking down data, processes, or problems into smaller, manageable parts




\subsection{Pattern Recognition}

\subsection{Algoritm Design}


\end{document}

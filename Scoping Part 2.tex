\documentclass{article}
\usepackage{geometry}
\usepackage[utf8]{inputenc}
\usepackage{graphicx}
\graphicspath{ {images/} }

\usepackage{geometry}
\geometry{a4paper, left=25mm, right=25mm, top=25mm, bottom=25mm}
\setlength{\parindent}{2em}
\setlength{\parskip}{1em}
\renewcommand{\baselinestretch}{1.3}


\usepackage{fancyhdr}
\pagestyle{fancy}
\fancyhf{}
\rhead{Sheriden Goldie - 42611814}
\lhead{FOAR705 - Digital Humanities}
\lfoot{Session 2 2019, Macquarie University}
\rfoot{Page \thepage}

\title{Proof of Concept - Scoping Exercise Part 2}
\author{By Sheriden Goldie}
\date{}

\begin{document}

\maketitle

\tableofcontents
%creates a table of contents based on sections and subsections

\pagebreak
%ends text on current page, and moves all following text to the next page.

\section{Identification}
In the process of completing a Business Analysis (Scoping Exercise 1) on my research process, I identified some key pain areas for further investigation.

These were:
\begin{itemize}
    \item Note taking and note management
    \item Formatting, and ability to change format for purpose (without damaging content)
    \item Reference and citation management
\end{itemize}

\subsection{Problem Identification}
The largest and most time consuming of these pains is ``Note taking and note management.''

The key aspects of this problem are:
\begin{itemize}
    \item Recording notes in a meaningful and organisable way
    \item Documentation of notes - recording note metadata (that being the source material, page numbers, date recorded etc)
    \item Recording and organising personal notes/responses to other notes
    \item creating networks of meaning from notes - mapping the key terms that link notes (or identifying notes that don't link to others)
    \item being able to access these notes and their metadata easily during final composition to enable to process of writing to be more flowing.
\end{itemize}

Even as I write these notes on the problem I can see there are certain points that need to be addressed in expressing this problem. 
\begin{itemize}
    \item What do I mean by a ``note''?
    \item Is there terminology that can express my ideas better?
    \item What are the different layers of data that I will be working with?
\end{itemize}

\section{Clarification}
\subsection{Key Project Terms}

To clarify this problem I am proposing to use the following terms:
\begin{itemize}
    \item Source text - the written document that information is taken from directly
    \item Quote - a direct transcription from a source text - a key data point.
    \item Source Text Metadata - the information necessary to identify the source text - enabling another person to find that text. Organised in such a way as to be useful for creating in-text citations and reference lists.
    \item Note - text written by the researcher reflecting on the content of a quote - A key data point.
    \item Summary Note - an extended note that might refer to two or more Source Texts and other Notes. 
    \item Tag - a unique category of metadata that is attached to a Quote or Note that enables it to be linked to other quotes or notes. Being able to render these links visually wold be nice. 
    \item Mapping - the process of finding connections and links between notes and quotes
\end{itemize}

\section{Problem Breakdown}
Considering this problem in terms of computational thinking, I will attempt to apply a practice of Decomposition, Pattern Recognition, and Algorithm Design to the problem of note taking and note organisation I have outlined. Though at this stage the latter practice of Algorithm design will not be a fully realised articulation.

\subsection{Decomposition}
Breaking down data, processes, or problems into smaller, manageable parts

The process of absorbing data for my research project is likely to follow these steps:
\begin{enumerate}
    \item Read Source Text
    \item Writing/recording an interesting passage or quoted idea from the Source Text. This is being referred to now as a Quote.
    \item Writing/recording my own thoughts and ideas in relation to the Quote - especially in relation to he current research project.
    \item Repeat these steps for numerous Source Texts
    \item Writing/recording extended Notes which might synthesise ideas into my own words - or that draw links between two or more Source Texts
    \item Collate all Quotes and Notes for review
    \item Find connections between Quotes and Notes - Represent this visually if possible
    \item Use connected groupings to record more Summary Notes
    \item Use the Networked data to create an essay/project outline
    \item Have the Quotes, Notes, Summary Notes, and Source Text Metadata available and easily accessible during writing to allow it to be integrated without interrupting the writing flow. 
    \item Use Source Text Metadata to create relevant in-text citations, as well as end of text reference list/bibliography (in the relevant format).
\end{enumerate}

\subsection{Pattern Recognition}
Observing patterns, trends, and regularities in data

In the process I outlined above - there are some processes that seem to be recurring. 

\begin{itemize}
    \item Writing/Recording - including relevant metadata. There are aspects of this that are repetitive like noting the metadata, but the recording of Notes is a cognitive and creative process - which resists digitisation. 
    \item Creating a network of key ideas/key terms (tags) between Notes and Quotes. This is a labour intensive task - and can very easily go haywire, especially if one is using a manual process like post-it notes or various notebook pages. 
    \item Organising data in a way that is accessible and navigateable. This is a task that could definitely be digitised - though it may require adjustments to general work practices. These adjustments could be beneficial in the long run, though challenging to implement.
\end{itemize}

\subsection{Algorithm Design}
Developing the step by step instructions for solving this and similar problems.

A key issue of this problem that I can apply algorithm design to is organising the data in a way that is accessible and navigateable.

This would require some set up and factors/requirements to consider would be:
\begin{itemize}
    \item Entering data into a digital format.
    \item Data is entered with sufficient metadata, like tags, in a frame of reference that can be abstracted from the core data.
    \item Creating an output framework for the secondary networked data to be displayed with.
\end{itemize}

To collate the data
\begin{enumerate}
    \item Using a digital format to aid the entering of data. 
    \item Enter data in a way that is linked to Tags, and Source Text Metadata
    \item Organise the data using Tags, and Source Text Metadata.
\end{enumerate}

To output a visual arrangement of notes and quotes - showing the connections:
\begin{enumerate}
    \item Data is organised using tags (categories - that are established using a controlled vocabulary)
    \item we are able to group data based on their tags
    \item these networks are able to be exported to a program that can visualise these links
    \item the information in the visualisation is able to link back to the main data/core information
\end{enumerate}

\end{document}
